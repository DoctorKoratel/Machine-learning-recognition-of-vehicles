\documentclass[a4paper,14pt]{extarticle} %размер бумаги устанавливаем А4, шрифт 14пунктов
\usepackage[left=2cm,right=2cm,top=2cm,bottom=2cm,bindingoffset=0cm]{geometry} % Меняем поля страницы
\usepackage{setspace}
\onehalfspacing % Полуторный интервал

\usepackage[T2A]{fontenc}
\usepackage[utf8]{inputenc}

\usepackage[english,russian]{babel} %используем русский и английский языки с переносами
\usepackage{amssymb,amsfonts,amsmath,mathtext,cite,enumerate,float} %подключаем нужные пакеты расширений

\usepackage[pdftex]{graphicx, color}
\usepackage{subfigure}
\usepackage{color}
\usepackage{listings}

\usepackage{algorithm}
\usepackage{algpseudocode}

\usepackage{pdflscape}

\usepackage{longtable}

\usepackage{float}
\floatname{algorithm}{Листинг}

\DeclareGraphicsExtensions{.png,.pdf,.jpg,.mps,.bmp}
\graphicspath{} %путь к рисункам
\usepackage{bmpsize}

\usepackage[nooneline]{caption} \captionsetup[table]{justification=raggedleft} \captionsetup[figure]{justification=centering,labelsep=endash}

\begin{document} 
\renewcommand{\figurename}{Рисунок}
\renewcommand{\baselinestretch}{1.5}
\renewcommand{\abstractname}{{Аннотация}}

\section{Общие положения}
\hspace{\parindent} В дипломной работе рассматривается задача распознавания марки автомобиля по фотографии и, хотя на данный момент уже существуют решения в данной области, до сих пор остается не изученной проблема дообучения таких систем. Базовый подход заключается в применении методов машинного обучения (ML --- Machine Learning), таких как нейронные сети (NN --- Neural Network, или ANN --- Artificial Neural Network) и деревья принятия решений (decision tree).

\section{Машинное обучение}
\hspace{\parindent} Машинное обучение можно определить как математическую дисциплину, или набор математических, статистических, численных методов и различных подходов из теории вероятностей для решения задачи автоматического выделения информации (Information Extraction) и ее интеллектуального анализа (Data Mining) из неструктурированного набора входных данных. Более подробно --- ставится задача сопоставления заданному значению (объекту) некоторого действия (ответа), таким образом можно говорить об обобщении задачи аппроксимации функции. 

Машинное обучение активно используется при классификации данных, например, распределение новостных статей по рубрикам. Отдельное внимание уделяется обработке изображений, их автоматическому аннотированию~\cite{farabet2013learning} или выделению одного определенного объекта на изображении,  задача распознавания рукописного текста (OCR --- Optical Character Recognition).

Основные подходы машинного обучения при работе с изображениями --- это использование нейронных сетей и деревьев принятия решений. Оба метода занимаются выделением опорных признаков на обрабатываемой картинке, после чего по ним выполняют классификацию. Перед использованием построенной модели необходимо ее обучить на тестовом наборе данных, выработать базу признаков. Алгоритмы обучения решают задачу оптимизации, занимаются подбором неизвестных коэффициентов в математической модели описывающей конкретный метод машинного обучения. Процесс обучения требует большого количества вычислительных операций, что заставляет искать альтернативные подходы при работе с предметной областью, в которую могут добавляться новые классы объектов. Вместо полного переобучения модели используют алгоритмы частичной корректировки коэффициентов (fine tuning). Такой подход дообучения требует меньше времени, но связан с потерей качества, так как не добавляет в модель признаков объектов новых классов.

\section{Деревья принятия решений} 

\section{Нейронные сети}

\newpage
\section*{Используемые обозначения}

\newpage
\bibliographystyle{gost780u}  %% стилевой файл для оформления по ГОСТу
\begin{flushleft}
\bibliography{biblio}     %% имя библиографической базы (bib-файла) 
\end{flushleft}

\end{document}