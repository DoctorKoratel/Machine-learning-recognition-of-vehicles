\documentclass[a4paper,14pt]{extarticle} %размер бумаги устанавливаем А4, шрифт 14пунктов
\usepackage[left=2cm,right=2cm,top=2cm,bottom=2cm,bindingoffset=0cm]{geometry} % Меняем поля страницы
\usepackage{setspace}
\onehalfspacing % Полуторный интервал

\usepackage[T2A]{fontenc}
\usepackage[utf8]{inputenc}

\usepackage[english,russian]{babel} %используем русский и английский языки с переносами
\usepackage{amssymb,amsfonts,amsmath,mathtext,cite,enumerate,float} %подключаем нужные пакеты расширений

\usepackage[pdftex]{graphicx, color}
\usepackage{subfigure}
\usepackage{color}
\usepackage{listings}

\usepackage{algorithm}
\usepackage{algpseudocode}

\usepackage{pdflscape}

\usepackage{longtable}

\usepackage{float}
\floatname{algorithm}{Листинг}

\DeclareGraphicsExtensions{.png,.pdf,.jpg,.mps,.bmp}
\usepackage{bmpsize}

\usepackage[nooneline]{caption} \captionsetup[table]{justification=raggedleft} \captionsetup[figure]{justification=centering,labelsep=endash}

\begin{document} 
\renewcommand{\figurename}{Рисунок}
\renewcommand{\baselinestretch}{1.5}
\renewcommand{\abstractname}{{Аннотация}}

\begin{center}
\huge Отчет о проведенной НИРС
\end{center}
\renewcommand{\contentsname}{\centering Разделы}
\tableofcontents
\newpage

\section{Существующие подходы определения и классификации автомобилей}
\subsection[Использование трехмерной структуры и нейросети]{Использование трехмерной структуры и нейросети~\cite{wu2001method}} \label{3Dstruct}
\hspace{\parindent} Данный метод основан на представлении автомобилей в виде полигональной модели, по которой происходит выделение признаков, передаваемых на вход нейросети. Трехмерная структура описания автомобиля разделена на восемь частей, в каждой из которых выделяется опорная вершина. В качестве признаков используется расстояние между каждой парой опорных вершин. Дополнительно рассматриваются параметры колес автомобиля, их радиус и положение.

Авторы работ использовали трехслойную, полносвязную нейронную сеть с 30 входами (количество признаков) и 120 выходами ($6 \times 20$, 6 --- количество классов, 20 --- число выходных нейронов для каждого класса). Результат определялся поиском максимального значения в выходном слое. 

При обучении сети использовался метод обратного распространения ошибки. В качестве ошибочной, для улучшения сходимости, использовалась экспоненциальная функция вместо квадратичной. 

Обучающая выборка состояла из 500 изображений с дорожных камер. Классификация производилась на большие/небольшие грузовики и автомобили. Тестовая выборка включала в себя еще 300 изображений, из которых правильно классифицировано 91\%, ошибочно 4\% и в 5\% случаев система не смогла определить автомобиль на изображении.

\subsection[Deep Belief Network]{Deep Belief Network~\cite{wang2014vehicle}}
\hspace{\parindent} Развитие идеи использования рекуррентных нейронных сетей, которые тяжело обучать из-за наличия обратных связей. Рассматривается архитектура при которой внутри спрятанных слоев используется ограниченная машина Больцмана. 

Авторы рассматривают задачу определения автомобиля по фотографии сзади (как с камеры авторегистратора). Входной слой имеет размерность соответствующую размерности изображения (признаки --- пикселы). Выходов два --- классы автомобиль и не автомобиль.

Тестирование происходило на 735 изображениях, рассматривались модели с одним, двумя и тремя скрытыми слоями. Наилучшая точность была получена при использовании двух скрытых слоев --- более 96\%. Также была произведена оценка точности для других алгоритмов: искуственных и сверточных нейросетей, $k$ ближайших соседей и метода опорных векторов. Среди них наибольшая точность результата получилась при использовании сверточных нейронных сетей, почти 95\%.

\subsection{Использование цепочки классификаторов}
\hspace{\parindent} В работе~\cite{brehar2010pillars} ставилась задача определения боковых стоек автомобиля. Решение задачи представляло собой многослойную схему классификации, от общего к частному. Такой подход называют усилением простых классификаторов (boosting classifier). Первые слои предназначены для обработки входного изображения, определения направления движения, колес, выделение боковой части автомобиля и только после этого поиск боковых стоек на основе гистограммы ориентированных градиентов и геометрических моделей.

Обучение системы производилось на изображениях автомобильного потока, в тестовую выборку также включены негативные примеры не содержащие транспортных средств. Тренировочное множество состоит из 100000 изображений без автомобилей и 4000 с автомобилями. Тестовая выборка содержит 1000 положительных и 200000 негативных примеров. Итоговая точность 90\% для положительных примеров и 99\% для негативных.

\subsection{Байесовские сети} \label{BayesNet}
\hspace{\parindent} Имея возможность выделения признаков для дальнейшего решения задачи классификации можно воспользоваться байесовской сетью. Такой подход использовался в работе выделения автомобилей на фотоаэроснимках низкого разрешения~\cite{zhao2003car}. Основная проблема при решении этой задачи --- высокое количество шумов: тени, солнечные блики, кроны деревьев и прочее.

Авторы предложили метод, основанный на использовании дополнительной известной информации о времени и месте съемки. Эти данные и выделенные признаки объектов вместе попадают в байесовскую сеть, которая производит отсев таких неправильно определенных кандидатах как тени от настоящих автомобилей и объектов находящихся не на дорожном полотне. 

\subsection[Car-Rec]{Car-Rec~\cite{jang2011car}}
\hspace{\parindent} Еще одно предложенное решение для задачи классификации автомобилей, заключающееся в выделении признаков и их обработке с использованием деревьев поиска по заготовленной базе. Для определения регионов интереса на изображении используется алгоритм SURF~\cite{bay2008speeded}, основанный на интегральном представлении входной картинки. Разработанная система имеет точность более 90\%.

\section{Классификация без явного определения признаков}
\hspace{\parindent} Практически все перечисленные выше подходы для решения задачи определения или классификации автомобилей занимались выделением опорных признаков по которым строился ответ. В пункте~\ref{3Dstruct} использовалась заготовленная трехмерная модель, описывающая транспортное средство, а в~\ref{BayesNet} рассматривались физические известные факторы предметной области.

Однако, не всегда имеется возможность применить данный подход, так как в общем случае не понятно по какому принципу выбирать признаки для классификации. Кроме этого, не известно как с течением времени будет развиваться предметная область и какие еще признаки необходимо зарезервировать для дополнительных классов. 

При использовании нейросетей, признаки для классификации автоматически вырабатываются при обучении модели. В случае необходимости добавить новые классы в классификатор, производят его переобучение, в результате чего выводятся новые признаки. Также возможно произвести более быструю настройку, скорректировав весовые коэффициенты без перестроения признаков (дообучение сети).

\subsection{Сверточные нейронные сети}
\hspace{\parindent} В работе~\cite{sermanet2012convolutional} рассматривалась задача распознавания номерных табличек на домах. Использовалась традиционная архитектура сверточной нейросети: первый сверточный слой выделял 16 карт признаков со сверткой $5 \times 5$, второй --- 512, со сверткой $7 \times 7$, два слоя классификации по 20 нейронов каждый. Применялся многоступенчатый подход выделения признаков (слоям классификации были доступны все предыдущие карты признаков). Описанная модель реализована на фреймворке EBLearn~\cite{sermanet2009eblearn}.

Был произведен подбор оптимального параметра степени свертки для алгоритма выделения признаков Lp-Pooling. Наилучшая точность получена при использовании коэффициента $p=4$. Кроме этого, перед передачей изображения в сеть, вокруг него достраивалась рамка из двух пикселов с нулевым значением, чтобы отцентрировать первый этап свертки рамкой $5 \times 5$ на границе рассматриваемой картинки.

Набор данных состоял из трех частей: обучающей, тестовой и дополнительной выборках. Дополнительная выборка состояла из большого количества легких для классификации изображений, а обучающая содержала небольшое число более сложных примеров. Обучение происходило на 6000 примеров, выбранных случайным образом из обучающего множества ($2/3$) и из дополнительного ($1/3$). В результате, при тестировании была получена точность более 95\%.

\subsection{Глубокие сверточные нейронные сети}
\hspace{\parindent} Один из вопросов систем распознавания и классификации изображений --- на сколько большую предметную область можно охватить используя автоматические методы. В рамках состязания ILSVRC проводится попытка классифицировать огромный тестовый набор изображений (более миллиона) по тысяче категорий.

Авторы работы~\cite{krizhevsky2012imagenet} занимались созданием одной из самых больших сверточных нейросетей для участия в ILSVRC-2012. Она состояла из 5 сверточных и 3 полносвязных слоев, включающих 60 миллионов параметров и 650 тысяч нейронов. В работе говорилось о необходимости и важности большой размерности сети, так как пр попытке избавиться хотя бы от одного слоя происходило сильное падение качества классификатора. 

Для уменьшения вероятности переобучения использовался подход выбывающих нейронов~\cite{srivastava2014dropout}. В качестве активационной функции была выбрана модель параметрического выпрямляемого элемента ReLU~\cite{dahl2013improving}. При тестировании по мере оценки ошибки top-5 error была получена точность 84.7\%.

На соревнованиях ILSVRC-2014 была представлена еще большая сеть, включающая 27 слоев~\cite{szegedy2014going}. Команде GoogLeNet получилось добиться результатов в 6.67\% ошибок.

\subsection{Плиточные сверточные нейронные сети}
\hspace{\parindent}

\section{Обучение нейросетей}
\hspace{\parindent}

\subsection{Эффективное обратное распространение ошибки}
\hspace{\parindent}

\subsection{Использование графического процессора}
\hspace{\parindent}

\subsection{Дообучение сети}
\hspace{\parindent}

\newpage
\bibliographystyle{gost780u}  %% стилевой файл для оформления по ГОСТу
\begin{flushleft}
\bibliography{biblio}     %% имя библиографической базы (bib-файла) 
\end{flushleft}

\end{document}